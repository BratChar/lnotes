\documentclass{beamer}
%\usetheme{Copenhagen}

% common language and font settings

\usepackage[CJKchecksingle]{xeCJK}
\punctstyle{hangmobanjiao}

% Adobe Legacy
% \setmainfont{Adobe Garamond Pro}[Ligatures=TeX]
% \setCJKmainfont{Adobe Song Std}[BoldFont={Adobe Heiti Std},ItalicFont={Adobe Kaiti Std}]
% \setCJKsansfont{Adobe Heiti Std}
% \setCJKmonofont{Adobe Fangsong Std}

% Adobe Source
\setmainfont{Source Serif Pro}[Ligatures=TeX]
\setsansfont{Source Sans Pro}
\setmonofont{Source Code Pro}
\setCJKmainfont{Source Han Serif SC}[BoldFont={Source Han Sans SC},ItalicFont={KaiTi}]
\setCJKsansfont{Source Han Sans SC}
\setCJKmonofont{FangSong}

% Google Noto
% \setmainfont{Noto Serif}[Ligatures=TeX]
% \setCJKmainfont[BoldFont={Noto Sans CJK SC},ItalicFont={KaiTi}]{Noto Serif CJK SC Light}
% \setCJKsansfont{Noto Sans CJK SC}
% \setCJKmonofont{FangSong}


\begin{document}

\begin{frame}
\title{金刚般若波罗蜜经}
\author{鸠摩罗什\ 译}
\date{}
\maketitle
\end{frame}

\frame{\tableofcontents}

\section{最大之乘,最正之宗}
\begin{frame}{最大之乘,最正之宗}
若菩萨有我相,人相,众生相,寿者相。即非菩萨。
\end{frame}

\section{自如之理,乃见真实}
\begin{frame}{自如之理,乃见真实}
\begin{block}{佛告须菩提}
凡所有相,皆是虚妄。若见诸相非相,则见如来。
\end{block}
\begin{alertblock}{佛告须菩提}
凡所有相,皆是虚妄。若见诸相非相,则见如来。
\end{alertblock}
\begin{exampleblock}{佛告须菩提}
凡所有相,皆是虚妄。若见诸相非相,则见如来。
\end{exampleblock}
\end{frame}

\section{修无为福,胜于布施}
\frame{\tableofcontents[currentsection]}
\begin{frame}{修无为福,胜于布施}
如恒河中所有沙数,如是沙等恒河,于意云何?是诸恒河沙,宁为多不?
\end{frame}

\section{受持此经,功德无量}
\begin{frame}{受持此经,功德无量}
\begin{itemize}
    \item 初日分以恒河沙等身布施
    \pause
    \item 中日分复以恒河沙等身布施
    \pause
    \item 后日分亦以恒河沙等身布施
    \pause
    \item 如是无量百千万亿劫以身布施
\end{itemize}
\end{frame}

\section{应现设化,亦非真实}
\begin{frame}{应现设化,亦非真实}
\transdissolve
一切有为法,如梦幻泡影,如露亦如电,应作如是观。
\end{frame}

\end{document}
