\chapter{再版序}

转眼间 lnotes 已经两岁了。在这两年间,它在雷友们的帮助下改正了许多缺点并茁壮地成长着。随着 \LaTeX 技术的进步,包老师感觉到零敲碎打缝缝补补已经不适应新时期的发展;lnotes 需要洗心革面重新做人,以求得人民群众的谅解。因为太懒,这个动手改版的日期被一再推迟。然而亡羊补牢,亦未为迟。

\section*{新版改进}
新版本力求实现以下几个目标:

\begin{compactenum}
    \item 系统性;结构完整,脉络清晰。
    \item 层次性;详略得当,重点突出。
    \item 进步性;技术先进,内容创新。
    \item 一致性;前后呼应,风格统一。
\end{compactenum}

古人云:知易行难。古人又云:法乎其上,得乎其中。为实现这些目标,本文作出以下具体调整:

\begin{compactenum}
    \item 全文按出版业惯例分为三大部分:前置部分,包括封面、标题、版权、献辞、目录、序、致谢等;主体部分包括九章;后置部分包括跋、附录、索引等。
    \item 第一章简介,历史回顾部分有较大扩充。
    \item 第二章入门,结构有所调整,重组了对齐和间距、特殊段落、列表等节,内容也有增强。
    \item 第三章字体,介绍电脑字体相关概念,字符集、编码、字体格式等,以及字体在 \LaTeX 中的应用。由于 \XeTeX 日趋成熟,新字体技术方案得到广泛应用;从前的一些中文解决方案和字体安装配置方法已经过时。本章由原第七章中文和第八章字体压缩合并而来,增加了 \XeTeX 相关内容。
    \item 第四章数学,结构基本不变,补充了部分示例的代码。
    \item 第五章插图,介绍图形格式及其转换,怎样插入图形,矢量绘图。增加了色彩模型介绍;若干小节标题略作调整,更加一致。
    \item \MP, PSTricks, PGF等绘图工具的介绍因篇幅较长,拆分为三个独立章节,内容亦有所增强和扩充。
    \item 第九章表格,增加了数字表格和宽表格两节,宽度控制和彩色表格有所增加和扩充。
    \item 第十章结构,介绍文档结构,标题、目录、长文档、参考文献、索引、超链接等。
    \item 第十一章布局,介绍页面尺寸,页面样式以及左右标记、章节标记等的定制,多栏、分页等。
    \item 第十二章应用,包括幻灯、书信、简历、棋谱等内容。
    \item 附录A软件和宏包。
    \item 附录B印刷简史,有图有真相。
    \item 索引,包括人物,学校、研究所、公司、政府部门等组织机构的索引。
\end{compactenum}

\section*{体例}

\begin{compactenum}
    \item 人物,西方人名正文中一般用原文,索引中加中文翻译。中国人和日本人正文中用中文,索引中加英文翻译。正文中人名第一次出现时附生卒年份。
    \item 组织、机构、公司、学校等,著名的正文中用中文,比如惠普、微软,索引中加英文。缩写,著名的正文中直接用,比如 MIT, IBM, ISO,索引中有全名;一般的先写中文名,括号内提供英文名和缩写。
    \item 人物和组织机构等,与本文没有直接关系的不加索引。
    \item 英文大小写,组织、机构、公司、学校等,一般首字母大写;其他词汇全部小写。
    \item 标点符号,中文间用全角,西文间用半角。
    \item 命令行程序、宏包、\LaTeX{}命令和环境等用等宽字体。
\end{compactenum}
